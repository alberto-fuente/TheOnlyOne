\capitulo{6}{Trabajos relacionados}

Existen varios videojuegos de naturaleza similar al proyecto y que han sido tomados como referencia en algunos aspectos durante el desarrollo.

Los videojuegos ``battle royale'' comparten características muy similares en cuanto a mecánicas principales, pero cada videojuego aporta su propia identidad alrededor de ellas, con estilos y funcionalidades distintos en cada caso que hacen que la experiencia de juego sea única.

A continuación se presentan tres videojuegos de este género que han servido de inspiración para este trabajo \cite{wiki:BestBattleRoyale}.

\subsection{Fortnite}
Fortnite es un videojuego creado por la empresa de videojuegos Epic Games que contiene dos modos de juego, uno cooperativo y el modo \textit{battle royale}. Realmente, este juego fue quien dio a conocer este término en la industria de los videojuegos, despertando la curiosidad tanto de otros desarrolladores como de jugadores de conocer en qué consistía este nuevo género. La novedad que supuso en la época, sumado a su estética y a su modelo de negocio ``Free to play'', llevó a Fortnite al éxito, hasta tal punto de convertirse en uno de los videojuegos más populares alrededor del mundo.

Si bien este no es un juego con una perspectiva en primera persona, sí comparte otras similitudes con el trabajo desarrollado: Tiene una estética estilizada, caricaturesca y colorida, que llama mucho la atención, especialmente en el público más joven, algo que en parte se ha tomado como referencia para definir la estética del proyecto. Algunos otros elementos que se han tomado como referencia son la estructura visual del inventario, su sistema de rareza de armas o las plataformas de salto que hay en el juego para impulsarse en el aire.

Una mecánica que destaca en este juego y que lo hace diferenciador del resto de juegos del mismo género es que el jugador tiene la capacidad de construir todo tipo de estructuras. Paredes, rampas o suelos son algunas cosas que puede colocar alrededor de él para refugiarse, cubrirse de los enemigos o llegar a lugares elevados fácilmente.

Además, como es propio de este género, en cada partida participan 100 jugadores, y tiene un elemento, la ``tormenta'', que hace daño a los jugadores, delimitando la zona segura del mapa que se va reduciendo y cambiando de lugar durante la partida.

\subsection{PUBG}
PUBG: Battlegrounds es un juego de la compañía PUBG Studios que también pertenece al género \textit{battle royale}. En este caso, pretende conseguir una experiencia más realista, con armas y equipamiento que existen en la vida real, con complementos intercambiables como miras, cargadores, modos de disparo o silenciadores.

Su diseño y estética también están enfocados en una línea realista, tratando de imitar a la realidad con los efectos visuales y de sonido, gráficos detallados, y un mapa muy amplio con elementos verosímiles. Este juego está más orientado para los jugadores que buscan una experiencia de juego más pura y fiel a la realidad.

Al ser un \textit{battle royale}, en cada partida participan 100 jugadores y el mapa va reduciendo su área segura con el tiempo hasta que solo quede un jugador.
Cabe destacar que el modo de juego puede alternarse entre primera y tercera persona según la preferencia del jugador.

\subsection{Apex Legends}
Apex Legends es un videojuego también \textit{battle royale}, desarrollado por Electronic Arts que de alguna manera combina elementos de los juegos descritos anteriormente. Tiene un tono ciertamente desenfadado, con una estética futurista y algo estilizada, junto con elementos más realistas como el estilo de las armas o un sentido de la estrategia algo mayor que en Fortnite por ejemplo.

Un elemento característico de este popular juego es que los jugadores deben elegir una ``leyenda'' al iniciar una partida. Las leyendas son distintos tipos de personajes, cada uno con habilidades y ventajas propias que le ayudarán durante la partida.

Al igual que en el resto de títulos del género, los jugadores deben situarse en una zona determinada del mapa para no recibir daño. Tiene mecánicas de movilidad interesantes como el poder deslizarse por largas colinas o agarrar tirolinas y lanzaderas para desplazarse más rápidamente por el mapa.

Dos elementos que han servido de inspiración para el trabajo son las cajas de suministros que hay repartidas por el mapa con objetos de interés para el jugador, y el \textit{HUD} flotante que aparece junto a un enemigo cuando se le inflige daño con el color y el número apropiados.
