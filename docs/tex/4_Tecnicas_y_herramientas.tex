\capitulo{4}{Técnicas y herramientas} \label{TecnicasYHerramientas}

En este apartado se van a nombrar una serie de técnicas y herramientas que han sido de gran utilidad para llevaar a cabo el proyecto con un flujo de trabajo adecuado y eficiente.

\subsection{Blender}
Blender\cite{wiki:Blender} es una suite de creación 3D multiplataforma gratuita y de código abierto con la que se pueden crear modelos, visualizaciones y animaciones 3D. Su elemento principal de creación es la malla o \textit{mesh}, una estructura basada en polígonos editables y moldeables para construir los volúmenes deseados. Se pueden añadir, mover, eliminar o modificar tanto las caras como los vértices que las forman para crear los modelos tridimensionales con precisión.

Otros términos asociados a esta herramienta son los mapas UV, materiales, \textit{shaders} o los modificadores, importantes para obtener un flujo de trabajo adecuado en esta herramienta.

En este proyecto se ha utilizado Blender para crear varios modelos 3D incluidos en el juego, así como para texturizar algunos de ellos y crear animaciones para otros.

\subsection{Firebase Realtime Database}
Firebase Realtime Database \cite{wiki:Firebase} es una sistema gestor de base de datos (SGBD) NoSQL creado por Google. La base de datos se utiliza de manera remota en la nube y sincroniza los datos en tiempo real con todos los clientes conectados. Su funcionamiento se verá más en profundidad en el anexo C.

Es la base de datos elegida para el proyecto para que los usuarios puedan guardar su progreso en el juego, así como registrarse e iniciar sesión en él. 

\subsection{Github}
Github \cite{wiki:Github} es una plataforma para alojar proyectos software que utiliza el control de versiones \textit{Git}. Está pensada para que los desarrolladores puedan subir el código de sus aplicaciones a los llamados repositorios, lugares virtuales alojados en la nube donde poder almacenar cualquier tipo de archivo para mantener el proyecto a salvo y poder gestionarlo de manera más eficiente y segura.

El proyecto utiliza Github a modo de repositorio, como gestor de versiones para llevar un control sobre el mismo, y como gestor de tareas junto con la metodología \textit{Kanban}. Ha sido una herramienta esencial para llevar a cabo un desarrollo eficiente del proyecto.

\subsection{Latex}
Latex \cite{wiki:Latex} es un procesador de textos orientado a la redacción de textos de alta calidad tipográfica, como libros académicos, papers científicos, artículos y tesis. Esta herramienta, en conjunto con el editor online \textbf{\textit{Overleaf}} \cite{wiki:Overleaf}, se ha utilizado para redactar tanto la memoria como los anexos del proyecto. Al ser un procesador de textos un tanto particular y poco intuitivo en primera instancia, \textit{Overleaf} resulta de gran ayuda para visualizar el resultado en tiempo real de lo redactado en formato \textit{pdf}.

\subsection{Audacity}
Audacity \cite{wiki:Audacity} es un software de código abierto para la edición y mezcla de pistas de audio. En este proyecto, su uso se ha enfocado en la mezcla de diferentes sonidos para crear efectos interesantes y adecuados al juego. También se ha usado para el postprocesado de estas mezclas, añadiendo diferentes filtros o efectos y para ajustar su volumen.

\subsection{Illustrator}
Adobe Illustrator \cite{wiki:Illustrator} es un software de edición de imágenes centrado en la creación, edición y composición de gráficos vectoriales, muy útil para crear logotipos, iconos, y demás elementos formados a partir de vectores.
Este programa se ha utilizado en el proyecto para la creación de la mayoría de elementos de las interfaces del videojuego, como iconos, botones, paneles o fondos.

\subsection{Mixamo}
Adobe Mixamo \cite{wiki:Mixamo} es un software de servicio web dedicado a proporcionar modelos y animaciones 3D. Si bien dispone de modelos 3D de personajes en alta calidad, este programa se utiliza principalmente para agregar animaciones a modelos 3D humanoides de manera fácil. Para ello, utiliza técnicas de machine learning para dotar al modelo de un “esqueleto humano” a partir del cual crear las animaciones automáticamente.

Como se explica en el apartado “Aspectos relevantes del proyecto” \ref{Aspectos relevantes}, se ha utilizado esta herramienta para dotar de animaciones a los enemigos.

\subsection{Unity}
Unity 3D o simplemente Unity \cite{wiki:Unity}, es el motor de videojuegos empleado en el proyecto. Integra los diferentes componentes propios de un motor de videojuegos para gestionar aspectos como los gráficos, las físicas, la animación, la inteligencia artificial o el sonido. El lenguaje que utiliza para programar los \textit{scripts} es C\#.\\
Otras de sus características más destacadas son su amplio soporte a múltiples plataformas (PC, dispositivos móviles, consolas, realidad virtual, etc.), la gran comunidad de usuarios que posee, así como su amplia y detallada documentación.

Dentro de esta herramienta, existen diferentes elementos y conceptos que se utilizan de manera recurrente a lo largo del desarrollo del proyecto que son importantes de entender para comprender mejor este trabajo, y que son explicados más detalladamente en el apartado \ref{Conceptos teóricos}.

\subsection{Visual Studio}
Visual Studio \cite{wiki:VisualStudio} es el entorno de programación predeterminado para crear los scripts en Unity. Es un entorno de desarrollo integrado (IDE) muy completo, compatible con varios lenguajes como C\# y C++. Además, ofrece funciones de autocompletado y de depuración útiles para arreglar errores y carga automáticamente algunos de los \textit{namespaces} más utilizados en Unity. También permite crear diagramas de clases fácilmente a partir de las clases ya creadas.