\capitulo{1}{Introducción}

Los videojuegos son productos software enfocados al entretenimiento que simulan experiencias a través de una pantalla, y que, a diferencia de otras formas de ocio, requieren de la interacción activa del usuario a través de alguna forma de control, como un \textit{joystick}, un teclado y un ratón, o sus propias manos, para que sucedan eventos y acciones dentro del videojuego a gusto del jugador. De esta forma, el usuario se convierte en parte activa de la realidad virtual en la que está inmerso y puede sentir que tiene el control de lo que sucede mientras vive experiencias muy variadas.

Y es precisamente esa variedad que tienen los videojuegos en cuanto a su complejidad, calidad gráfica, temática, estilo artístico, género, historia narrativa, e incluso rango de precio, lo que ha hecho que en los últimos años se hayan convertido en un producto masivo, con el que un público cada vez más amplio puede disfrutar de todo tipo de experiencias jugables acorde a sus gustos y preferencias.

A esto se le suma el gran avance tecnológico que han sufrido recientemente los distintas dispositivos electrónicos, convirtiéndose en máquinas cada vez más potentes, lo que da como resultado que cualquiera pueda disfrutar de una partida a su juego favorito donde, cuando y como desee, ya sea en ordenadores, móviles, tabletas, o claro está, en las amadas consolas.

Los videojuegos tienen más alcance y popularidad que nunca. Se han convertido en una de las formas de entretenimiento más importantes y extendidas en todo el mundo, lo que también ha provocado que sea actualmente una de las industrias que más dinero genera.
Según estima el último informe de \textit{Accenture} \cite{wiki:videojuegos}, el valor total de la industria de los videojuegos está por encima de los 245.000 millones de euros y ha superado ya a las industrias del cine y la música juntas.

Teniendo en cuenta estos datos, no es de extrañar que la popularidad de los videojuegos no se haya visto incrementada sólo en el lado de los jugadores, sino también en el de los desarrolladores de software, que han visto una oportunidad para crear productos únicos que atraigan a todo tipo de personas.

No obstante, los videojuegos son productos muy particulares que involucran diferentes ámbitos, como la programación, el modelado, el diseño de interfaces, el arte, la música, el diseño de espacios, la animación o la capacidad narrativa, entre otros.                                                   
Es este sentimiento de desafío, unido a la curiosidad y a la pasión por los videojuegos la que ha llevado al alumno a querer entender el proceso de desarrollar  un videojuego desde la perspectiva del programador de software y los retos que ello conlleva.


\section{Materiales adjuntos}

El proyecto está formado por los siguientes materiales:
\begin{itemize}
\item Conjunto de archivos que forman la aplicación \textbf{``The Only One''} para Windows y Linux.
\item Repositorio de Github con los códigos fuente y resto archivos necesarios para su edición en el motor de Unity.
\item Vídeo explicativo del uso del videojuego.
\item Memoria del Trabajo de Fin de Grado
\item Anexos del proyecto
\end{itemize}

\section{Estructura de la memoria}

La memoria consta de los siguientes puntos a tratar:
\begin{itemize}
\item \textbf{Introducción.} Breve contexto y descripción del proyecto, así como los materiales y la estructura de la memoria.
\item \textbf{Objetivos del proyecto.} Enumeración de los objetivos principales que se persiguen con el proyecto.
\item \textbf{Conceptos teóricos.} Explicación de los términos y conceptos en relación a la materia que se trata, esenciales para comprender el desarrollo del proyecto.
\item \textbf{Técnicas y herramientas.} Descripción de las técnicas, herramientas y programas utilizados durante el desarrollo del proyecto.
\item \textbf{Aspectos relevantes.} Explicación de los aspectos más interesantes que tuvieron lugar durante el desarrollo del proyecto, así como las cuestiones y dificultades más destacables que surgieron y cómo se abordaron.
\item \textbf{Trabajos relacionados.} Breve descripción de trabajos y productos de similar naturaleza al proyecto que se expone.
\item \textbf{Conclusiones y Líneas de trabajo futuras.} Exposición de las conclusiones obtenidas una vez finalizado el desarrollo del proyecto, así como los pasos a seguir en el futuro para la mejora del producto desarrollado.
\end{itemize}